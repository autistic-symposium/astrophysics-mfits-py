
\section{Introduction}


{\it Neutron stars} (NS)  are the densest astrophysical objects
with a physical boundary \cite{THORN}. These objects are born when a massive star undergoes a core collapse supernova explosion. Within only tens of seconds after their birth, they are said to be cold, \ie their temperatures are far below the {\it Fermi energy} of their constituent particles. At this point, their matter is accurately described by the one-parameter equation of state that governs the {\it cold dense  matter} above nuclear density (\ie the mass and radius of the star only depend on the central density) \cite{LAT10}.  The {\it Tolman-Oppenheimer-Volkov} (TOV) relativistic stellar structure equations describes the  extremely dense cold matter in their cores by relating {\it pressure} and {\it energy density} (the equation of state, EoS hereafter). There is a unique map between the microscopic pressure-density relation ($P-\rho$) and the macroscopic mass-radius relation ($M-R$) \cite{LIN92} and these equations are constrained if the neutron stars {\it masses} and {\it radii} are determined by astrophysical methods \cite{LM07}. 



\quad

While precise {\it mass} measurements of neutron stars have been possible for a long time using techniques such as {\it pulsar timing} \cite{LO08}, the development of techniques to simultaneously achieve their radii are still under way. The fundamental properties of any star is based on a close study of the {\it emission spectrum} emerging from their {\it photospheres}. The most prominent candidates for such measurements are (thermonuclear) {\it bursting neutron stars} (X-ray bursters, XRB) in {\it quiescent low mass X-ray binary systems} (qLMXBs) that present {\it photospheric radius expansion} (PRE) \cite{LE93}. In this paper we review the state art of the X-ray Bursters problem and their implications to constrain to the equation of state of the cold dense matter.

\quad



%%%%%%%%%%%%%%%%%%%%%%%%%%%%%%%%%%%%%%%%%%%%%%%%%%%%%%%%%%%

%%%%%%%%%%%%%%%%%%%%%%%%%%%%%%%%%%%%%%%%%%%%%%%%%%%%%%%%%%%

%%%%%%%%%%%%%%%%%%%%%%%%%%%%%%%%%%%%%%%%%%%%%%%%%%%%%%%%%%%

%%%%%%%%%%%%%%%%%%%%%%%%%%%%%%%%%%%%%%%%%%%%%%%%%%%%%%%%%%%

%%%%%%%%%%%%%%%%%%%%%%%%%%%%%%%%%%%%%%%%%%%%%%%%%%%%%%%%%%%

%%%%%%%%%%%%%%%%%%%%%%%%%%%%%%%%%%%%%%%%%%%%%%%%%%%%%%%%%%%





\subsection{Thermonuclear X-ray Bursters (XRB)}

{{\it Binary X-ray pulsars} were discovered in the seventies and were thought as  neutron stars undergoing {\it accretion from a companion} \cite{WLP80}. Rosenbluth et al. \cite{RO73} pointed out that the nuclear energy coming {\it outward from the surface layers} (due the nuclear burning in the interior layers) would be independent of the source of energy that was being radiated directly from the neutron star's {\it photosphere} (due the burning of the accreted material and the strong surface gravity on these particles). 

\quad

Hansen et al. \cite{HVH75} showed that the nuclear burning of these accreted particles should be {\it unstable} and lead to {\it thermonuclear flashes}. This would be caused by  unstable burning of a several meters thick layer of {\it accreted hydrogen and helium} on the surface of neutron stars from the binary companion.  The group noted that the energy of these flashes could produce variable X-ray emission from the neutron star, but with shorter times than the time scales of the known X-ray pulsars at that time. 

  
\quad

Nuclear  fusion within the accreted layer on the neutron star provides energies around $\epsilon_{\rm thermo} \sim 5$ MeV per nucleon (when the solar mix goes to the iron group elements) \cite{BI00}. On the other hand, in-falling particles from accretion (\ie the gravitational energy per accreted baryon of proton mass $m_p$) release energies around
$\epsilon_{\rm grav} = \frac{GM m_p}{R} \sim 200$ MeV per nucleon. This energy is liberated in form of kinetic energy, gamma-ray photons (emerging in the X-ray burst),  and neutrinos. The time that takes for heat transport to cool the deep envelope (thermal time) is around seconds, which is much shorter than the time needed  to accumulate the  accreted material (hours to days, since the compression is adiabatic). If the accreted fuel was burned at the rate of accretion, any evidence of nuclear physics would be swamped by  the gravitational energy. The only way the nuclear energy can be seen is because the fuel is stored for a long period and then rapidly burned. In other words, even though the thermonuclear energy is one order  of magnitude smaller than the gravitational energy, it is released all together in a very energetic burst, powering the X-ray Bursters \cite{SB03}.

 

\quad


Thermonuclear X-ray bursts are a special class of  members of {\it quiescent low mass X-ray binary systems} with a relatively low accretion rate, showing quasi-periodical thermonuclear flashes on their surfaces. Binary systems in which the mass of the donor star is less than $\sim 1 M_{\odot}$ are defined as qLMXBs \cite{WLP80}. In general, these systems are concentrated towards the galactic center and in locations outside regions of active star formation, such as globular clusters (sometimes allowing an independent measurement of their distances). Thermonuclear X-ray bursts are a common phenomenon in qLMXBs: up to date $\sim 100$ sources have been observed \cite{LIST}. 

\quad



All the mechanisms on the X-ray burst phenomena utilize accretion of matter onto a collapsed object (\ie degenerate white dwarf, neutron star, or black hole). In every instance, the collapsed object serves as one of two functions, in accord to the following classification \cite{WLP80}:

\begin{description}
\item[Type-I (thermonuclear) bursts:] The energy source is the {\it nuclear energy}, invoking thermonuclear flashes in the surface layers of an accreting neutron star. The also show the following proprieties:
\begin{itemize}
\item the rise times of the light curves are shorter than the decay times, with both lasting a total of seconds to minutes;
\item the burst profiles are shorter at higher energies;
\item  the burst profiles show an {\it exponential-like rise} and a {\it thermal (blackbody) spectrum
during the decay}, moreover there is a  distinct {\it spectral softening} (cooling) observed during the decay.
\end{itemize}

\item[Type-II (accretion instabilities) X-ray busts:] The energy source is the {\it gravitational potential energy}, invoking instabilities in the accretion flow onto a collapsed object. The spectrum is also thermal with no significant cooling during the decay.

\end{description}
 


\quad



%%%%%%%%%%%%%%%%%%%%%%%%%%%%%%%%%%%%%%%%%%%%%%%%%%%%%%%%%%%

%%%%%%%%%%%%%%%%%%%%%%%%%%%%%%%%%%%%%%%%%%%%%%%%%%%%%%%%%%%

%%%%%%%%%%%%%%%%%%%%%%%%%%%%%%%%%%%%%%%%%%%%%%%%%%%%%%%%%%%

%%%%%%%%%%%%%%%%%%%%%%%%%%%%%%%%%%%%%%%%%%%%%%%%%%%%%%%%%%%

%%%%%%%%%%%%%%%%%%%%%%%%%%%%%%%%%%%%%%%%%%%%%%%%%%%%%%%%%%%

%%%%%%%%%%%%%%%%%%%%%%%%%%%%%%%%%%%%%%%%%%%%%%%%%%%%%%%%%%%

%%%%%%%%%%%%%%%%%%%%%%%%%%%%%%%%%%%%%%%%%%%%%%%%%%%%%%%%%%%

%%%%%%%%%%%%%%%%%%%%%%%%%%%%%%%%%%%%%%%%%%%%%%%%%%%%%%%%%%%

%%%%%%%%%%%%%%%%%%%%%%%%%%%%%%%%%%%%%%%%%%%%%%%%%%%%%%%%%%%

%%%%%%%%%%%%%%%%%%%%%%%%%%%%%%%%%%%%%%%%%%%%%%%%%%%%%%%%%%%








\subsection{The Spectra of Thermonuclear X-ray Busters}

\subsubsection*{The Blackbody Radius}

Depending on the X-ray bursts physical conditions (\eg accretion rate, composition of the fuel), thermonuclear X-ray bursters can have recurrence times between minutes and weeks. During the intervals between accretions, the {\it observed emission} would come directly from the {\it neutron star surface} \cite{VAN90}. Supposing that the entire surface of the X-ray burster is being observed,  a combination of spectroscopic data where the  {\it measured flux} is compared to the {\it flux emerging} from the stellar photosphere could give  the stellar {\it radius}.   

\quad

A typical spectral analysis of an X-ray burst is made by calculating the temporal variation  of the blackbody temperature, $T_{\rm bb}$, the bolometric flux, $F$, and the blackbody radius, $R_{\rm bb}$. The radius of such a blackbody emitting a flux at a distance $D$, is
\begin{equation*}
R_{\rm bb} = D\Bigg ( \frac{F}{\sigma_{\rm SB} T_{\rm bb}^4} \Bigg)^{1/2},
\end{equation*}
where $\sigma_{\rm SB}\approx 5.6704 \times 10^{-5}\ \textrm{erg}\,\textrm{ cm}^{-2}\,\textrm{s}^{-1}\,\textrm{K}^{-4} $ is the {\it Stefan-Boltzmann constant}. If the distance is known, with measurements of the bolometric fluxes and the blackbody temperatures this quantity can be calculated \cite{BI00}. During burst decay, in a diagram of $\log F$ versus $\log kT_{\rm bb}$, one finds that $R_{\rm bb}$ is approximately constant (\ie the cooling track is a straight line) \cite{WLP80}.

\quad



\subsubsection*{The Surface Gravity}\label{sz}

As long as the neutron star atmosphere is geometrically {\it thin}, the general relativistic corrections to the relevant physics are negligible. The only gravitational effect that is important for the X-ray bursters problem is the neutron star {\it surface  gravity} \cite{MAD04}. When the photons travel away from the star, out of the star's {\it gravitational well}, they lose energy (\ie they are {\it redshifted}). When they are out in the {\it Euclidean} space their energy will be down by a factor of $1+z$. After thIs they will not lose any significant amount of energy as they travel to Earth.

\quad

The continuum spectral shape of X-ray bursts is sensitive to the  surface gravity. Their spectroscopic observations give the {\it atomic absorption spectrum} which through pressure broadening and general relativity effects are sensitive to  the  acceleration of  gravity at the stellar surface\footnote{These equations are valid for non-rotating neutron stars ({\it Schwarzschild metric}), however they are a good approximation for neutron star rotation periods larger than few seconds.},
\begin{equation}
g = \frac{GM}{R^2} (1 +z),
\label{g}
\end{equation}
 where the gravitational {\it redshift} is
\begin{equation}
 1 +z = \Bigg(1- \frac{2GM}{c^2R} \Bigg)^{-1/2} = \Bigg( 1 - \frac{R_s}{R} \Bigg)^{-1/2},
\label{z}
\end{equation} 
 and $R_{\rm s}$ is the {\it Schwarzschild radius} of the neutron star, $R_{\rm s} = 2GM/c^2$.

\quad


If a {\it discrete feature} (a spectral line) were present and identified in a burst spectrum, one would have a direct measurement of the gravitational redshift of the neutron surface and thus the ratio $M/R$. By relating the Eqs. \ref{g} and \ref{z}, one could obtain $M$ in function of $g$ and $R$ \cite{LM07} \cite{MAD04},
\begin{equation}
M= \frac{g^2R^3}{c^2G}\Bigg[\Big(1 + \frac{c^4}{g^2R^2}\Big)^{1/2} -1 \Bigg].
\label{m3}
\end{equation}

\quad

On another hand, a simultaneously measurement of $z$ and 
\begin{equation*}
R = R_{\infty} (1+z)^{-1},
\end{equation*}
would complete solve  Eq. \ref{m3},
\begin{equation*}
M= \frac{c^2}{2G} R_{\infty} (1+z)^{-1}[1-(1+z)^{-2}],
\end{equation*}

\quad

This is the most reliable way to overcome the systematic uncertainties in the interpretation of the continuum spectra from bursts. However,  the quantity $R_{\infty}/D$ that is observationally determined, where $D$ is usually uncertain.


\quad

Furthermore, the proprieties of the emitted radiation as seen by an observer on the neutron star surface, including bolometric luminosity and the effective and color temperatures, will be different from those measured by a distant observer. They are also  determined from the gravitational redshift at the neutron star surface. The luminosity and the temperature of a blackbody emitter observed locally at the surface of the neutron star and from a distance observer are 
\begin{equation*}
L_{\infty} = L (1+z)^{-2},
\end{equation*}
and
\begin{equation*}
T_{\infty} = T (1+z)^{-1}.
\end{equation*}

\quad 





\subsubsection*{The Photospheric Radius Expansion (PRE)}

The {\it Eddington limit} is defined by  the maximum luminosity in which the radiation force outwards balances the gravitational force inwards in hydrostatic equilibrium.  In these cases, the X-ray bursts lead to  {\it photospheric radius expansion} (PRE)  and the phenomenon is a  prominent tool to  simultaneously measure the neutron star {\it masses and radii} \cite{VAN90} \cite{DAM90}. 

\quad

PRE events occurs when the thermonuclear burning at the bottom of the freshly accreted matter can become so powerful that  {\it local luminosity} gets close to the Eddington limit, 
\begin{equation}
L_{\rm Edd} =  \frac{4 \pi G M c}{\kappa_{\rm e}}(1+z),
\label{ledd}
\end{equation}
where $z$ is given by Eq. \ref{z} and the {\it Thomson scattering opacity} is given by
\begin{equation*}
 \kappa_{\rm e} = \sigma_{\rm T} \frac{\rm N_e}{\rho} \sim 0.2 (1 + X) \mbox{ cm}^2 \mbox{ g}^{-1},
 \label{kappa}
\end{equation*}
with $\sigma_{\rm T} = 6.65 \times 10^{-25}$ cm$^2$ (Thompson cross-section), $\rho$ the gas density, $N_{\rm e}$ the electron number density,  and $X$ is the hydrogen mass fraction by mass of the photospheric matter. The last term of this equation is exact if the opacity is dominated by electron scattering. 

\quad



The total flux emission can be approached to the Eddington (flux) limit  if the matter spreads all the way from the equator to the poles. In this case, the spectra present peak sources in the same order of magnitude to the Eddington flux,
\begin{equation}
F_{\rm Edd} = \frac{L_{\rm Edd}}{4 \pi D^2} (1+z)^{-2}= \frac{GMc}{\kappa_{\rm e} D^2} (1+z)^{-1}.
\label{fedd}
\end{equation}


\quad


Moreover, when the PRE burst occurs, the  {\it photospheric radius increases} as a result of radiation pressure, while the {\it effective temperature, $T_{\rm eff}$, decreases}.  During the expansion and contraction phase of the photosphere, the local luminosity, as measured by a local observer on the photosphere,  remains extremely close to the Eddington luminosity and all the excess flux is  converted to kinetic and gravitation potential energy. 


\quad

The moment when the photosphere falls back to the neutron star surface after the PRE event is called {\it touchdown}, which is the moment when the {\it effective temperature}, $T_{\rm eff}$,  reaches its maximum. A small {\it apparent radius}, $R_{\rm app}$, at the touchdown does not  mean that the photosphere actually coincides with the neutron star surface, because at luminosities very close to Eddington limit the color correction can be rather large. If the touchdown actually meant that the photosphere returned to its original size then the flux would have to be slightly less than $F_{\rm Edd}$ (or the radiation pressure would continue to push the photosphere outward). In many analysis available in the literature, best fits has shown  that $F_{\rm touchdown} > F_{\rm Edd}$  \cite{LAT10} \cite{LAT12}.

\subsubsection*{The Atmospheric Color Correction  Factor} 


Although the burst spectra are observationally shaped by a blackbody (Planck) function, they should be {\it harder} than a blackbody at the {\it effective temperature} of the atmosphere,
\begin{equation}
T_{\rm Eff} = \Bigg[\frac{L}{4\pi R^2 \sigma_{\rm SB}} \Bigg]^{1/4},
\label{teff}
\end{equation}

\quad


Assuming  that the burst emission is isotropic (which may not be the case), we can  convert  flux directly from luminosity.  For an  uniform spherical blackbody emitter of radius $R_{\infty}$, at distance $D$, the luminosity seen by an observer at Earth is
\begin{equation*}
L_{\infty} = 4\pi R_{\infty}^2   \sigma_{\rm SB} T_{\infty, \rm Eff}^4 = 4\pi D^2 F_{\infty}.
\end{equation*}

\quad



 In this sense, PRE events provide information about the neutron star compactness: they give the observed Eddington flux ($F_{\rm Edd}$, with a distance dependent mass-radius relation),  and the maximum effective temperature of the surface (with a mass-radius relation). Both $T_{\rm eff}$ and $F_{\rm Edd}$  depend on the composition of the atmosphere, however a measurement of the first when the Eddington limit is clearly reached would give a constraint of the mass and radius of the neutron star independent of the distance of the source  \cite{BI00} \cite{SPW10}.

\quad

The {\it harder model spectra}  refers to the fact that the {\it high energy tails} of the model spectra always lie {\it above} the corresponding blackbody spectra, simply meaning that the spectrum has  {\it more high energy photons} than a {\it soft} one.  In practice terms, this is reflected on a {\it color correction factor}, $f_{\rm c}$, of referred model being larger than one.  The hardness of{\it observed/emergent spectra} is well-fitted by a {\it diluted blackbody function} with a {\it color temperature}, $T_{\rm bb}$,
\begin{equation}
 F_{\rm E} \sim \frac{1}{f_{\rm c}^4} B(T_{\rm bb}),
 \label{dilbb}
\end{equation}
with
\begin{equation*}
 f_{\rm c} =\frac{T_{\rm bb}}{T_{\rm Eff}} = (R_{\infty}{D})^2.
 \label{fc}
\end{equation*}


\quad



For instance, a first approximation of the color temperature factor suggests
$$f_{\rm c} \sim \frac{T_{{\rm upper layers}}}{T_{\rm Eff}} .$$

\quad

 {\it Compton scattering} is an inelastic effect of the atmosphere however it does not necessary causes the hardening. This effect can even make the spectrum be softer: in the {\it rest frame} of the electron, the photon loses energy making the stationary electron move. If we change to the the {\it lab frame} instead, the photon could either {\it downscater} or {\it upscatter}, depending on parameters such as the speed of the electron and the energy of the photon. The hardening of the X-ray spectra is actually  due to the  {\it strong frequency-dependent absorption of the atmosphere}.

\quad



Furthermore, an accurate correct evaluation of $f_{\rm c}$ depends on many parameters of the neutron star atmosphere, such as  chemical composition, luminosity, and surface gravity. 
Any of the spectral methods to determine the neutron stars radii from X-ray bursters are only possible with an accurate spectral models for the neutron stars  atmospheres, specially at  {\it low effective temperatures}.

\quad




\subsubsection*{The Apparent Area during the Cooling Phase}

  The emitting spectra during the decay (cooling) phase of the burst is  well-approximated by a diluted blackbody curve where the observed flux $F_{cool}$\footnote{ $F=F_{cool}$ is the bolometric flux and is not directly measured.} varies as the 4th power of the observed blackbody (color) temperature $T_{\rm bb}$.  Moreover, a second spectroscopic quantity that can be used to infer the neutron star mass and radius during the {\it cooling tails} is the (measured) {\it normalized surface area} of the emitting region,  which  remains {\it constant}  while the {\it flux} and the {\it blackbody  temperature decreases},
\begin{equation}
A=  \frac{  F_{\infty  }}{\sigma_{\rm SB} T_{\infty,\rm bb}^4} = \frac{F_{\infty  }}{\sigma_{\rm SB}(f_{\rm c} T_{\rm Eff})^4}= \frac{1}{f_{\rm c}^4}\frac{R^2_{\rm app}}{D^2} = \frac{1}{f_{\rm c}^4} \Bigg(\frac{R}{ D} \Bigg[\frac{\mbox{ km}}{10\mbox{ kpc}}\Bigg] \Bigg)^2  (1+z).
\label{A}
\end{equation}

\quad

This quantity also remains the same in recurring X-ray burst events of the same source. The {\it apparent radius} of the emitting region, $R_{\rm app}$,  can be interpreted as an indication of the fraction of the star that is covered by freshly accreted fuel.


\quad

If one could find discrete spectral features in a particular neutron star, as described in the Section \ref{sz}, the redshift of such object could be directly calculated. With this information,  Eq. \ref{A} would give all the information needed to uniquely identify  the mass and the radius of the neutron star \cite{LM07},
\begin{equation*}
M = \frac{c^5}{4G\kappa_{\rm e}} \frac{Af_{\rm c}^{-4}}{F_{\infty \rm Edd}} \Big[ 1 - (1+z)^{-2}\Big]^2 (1+z)^{-3},
\label{mlat}
\end{equation*}
and
\begin{equation*}
R = \frac{c^3}{2 \kappa_{\rm e}} \frac{ Af_{\rm c}^{-4} }{F_{\infty \rm Edd} }\Big [ 1 - (1+z)^{-2}\Big](1+z)^{-3}.
\label{rlat}
\end{equation*}

\quad 




\subsubsection*{The Blackbody Normalization}\label{sK}


The  redshift (Eq. \ref{z}) is usually difficult to be obtained without an accurate information of distance, and it is not possible to extract it from the surface gravity (Eq. \ref{g}) unless you already have the mass and the radius already. In more recent atmosphere models an alternative method of calculation is  suggested to fit the mass and radius of X-rays bursters without the needing the explicit value of $z$ \cite{SPW10}.  The distance-dependent quantities $A$ (Eq. \ref{A}) and $F_{Edd}$ (Eq. \ref{fedd}) can be combined to the distance-independent {\it Eddington temperature} which is the {\it apparent effective temperature} corresponding to $L_{Edd}$. To obtain the best-fit the following procedure could be tested:
\begin{enumerate}
\item guess a value for $A$ (Eq. \ref{A}) and $F_{\rm Edd}$ (Eq. \ref{fedd});
\item find which of the many model spectra best fits to the observed data, in terms of $l$ (Eq. \ref{ll}), $g$ (Eq. \ref{g}), and chemical composition (comparing the sets over the entire range of $l$ values);\item iterate the new values of $A$ and $F_{\rm Edd}$ until convergence.
\item Once you have the desired values of $A$ and $F_{\rm Edd}$, the mass and radius can be calculated if the distance to the source is known (for example, for sources in globular clusters), by plotting  three curves on the $M-R$ plane, corresponding to values of $A$, $F_{Edd}$, $T_{\infty, \rm  Edd}$.
\end{enumerate}


\quad

A quantity of interest for above calculation is the {\it blackbody normalization},
\begin{equation}
K = \Bigg ( \frac{R_{bb}}{D} \Bigg[\frac{\mbox{ km}}{10\mbox{ kpc}}\Bigg]  \Bigg)^2 = \frac{1}{f_{\rm c}^4} \Bigg(\frac{R (1+z)}{D}   \Bigg[\frac{\mbox{ km}}{10\mbox{ kpc}}\Bigg]    \Bigg)^2 = (Af_{\rm c})^{-4},
\label{K}
\end{equation}
where the evolution of $K^{-1/4}$ at late stages of the burst reflects the evolution of the color correction factor. 



\quad

Some mathematical manipulations give the following relations:

\begin{equation*}
T_{\infty, \rm Edd} = \Bigg(\frac{gc}{\sigma_{\rm SB} \kappa_{\rm e}} \Bigg)^{1/4} (1+z)^{-1} = 6.4\times 10^9 A^{-1} F_{\rm Edd}^{1/4} \mbox{ K},
\end{equation*}
with $\kappa_{\rm e}$ given by Eq. \ref{kappa}, $g$ given by Eq. \ref{g}, $z$ given by Eq. \ref{z}, $A$ given by Eq. \ref{A}, and $F_{\rm Edd}$ given by Eq. \ref{fedd}.

\begin{equation*}
F=\frac{R_{\rm bb}^2 \sigma_{\rm SB}T^4_{\rm bb}}{D^2} = \frac{ R^2\sigma_{\rm SB}T^4_{\rm eff} (1+z)^{-2}}{D^2},
\end{equation*}
 with
\begin{equation*}
T_{\rm bb} = f_{\rm c} T_{\rm eff} (1+z)^{-1},
\end{equation*}
giving
 \begin{equation*}
R_{\rm \infty} = R(1+z) = R_{\rm bb} f_{\rm c}^2.
\end{equation*}



\quad


Finally, we can define the star {\it compactness} as
\begin{equation*}
u = \frac{R_{\rm s}}{R} = 1 - (1+z)^{-2},
\end{equation*}
obtaining the relations for radius and mass:
\begin{equation*}
R = \frac{c^3}{2\kappa_{\rm e} \sigma_{\rm SB} T^4_{\infty, \rm Edd}} u (1-u)^{3/2},
\end{equation*}
and
\begin{equation*}
M = \frac{R}{2.95 \mbox{ km}}u M_{\odot}.
\end{equation*}


%%%%%%%%%%%%%%%%%%%%%%%%%%%%%%%%%%%%%%%%%%%%%%%%%%%%%%%%%%%

%%%%%%%%%%%%%%%%%%%%%%%%%%%%%%%%%%%%%%%%%%%%%%%%%%%%%%%%%%%

%%%%%%%%%%%%%%%%%%%%%%%%%%%%%%%%%%%%%%%%%%%%%%%%%%%%%%%%%%%

%%%%%%%%%%%%%%%%%%%%%%%%%%%%%%%%%%%%%%%%%%%%%%%%%%%%%%%%%%%

%%%%%%%%%%%%%%%%%%%%%%%%%%%%%%%%%%%%%%%%%%%%%%%%%%%%%%%%%%%

%%%%%%%%%%%%%%%%%%%%%%%%%%%%%%%%%%%%%%%%%%%%%%%%%%%%%%%%%%%

%%%%%%%%%%%%%%%%%%%%%%%%%%%%%%%%%%%%%%%%%%%%%%%%%%%%%%%%%%%

%%%%%%%%%%%%%%%%%%%%%%%%%%%%%%%%%%%%%%%%%%%%%%%%%%%%%%%%%%%

%%%%%%%%%%%%%%%%%%%%%%%%%%%%%%%%%%%%%%%%%%%%%%%%%%%%%%%%%%%

%%%%%%%%%%%%%%%%%%%%%%%%%%%%%%%%%%%%%%%%%%%%%%%%%%%%%%%%%%%






\section{X-ray Bursters Atmosphere Models}


\subsection{Brief History - Highlights}

In the past three decades, a range of neutron star atmosphere models were made available in the literature, showing the dependence of the {\it emergent  surface flux}\footnote{For some authors, the  flux is presented over $4\pi$, referenced as $H_E$.}  versus the frequency (or energy in keV) of the photons, and  basing their calculations on parameters such as {\it chemical composition} (\eg the hydrogen mass fraction $X$ and solar mixings), {\it surface gravity, $g$,} (Eq. \ref{g}), and  {\it relative surface luminosities}, related to Eqs. \ref{ledd} and \ref{teff} as
\begin{equation}
l=\frac{L}{L_{\rm Edd}} = \Bigg(\frac{T_{\rm Eff}}{T_{\rm Edd}}\Bigg)^4.
\label{ll}
\end{equation}

\quad

  In general, these models assume {\it plane-parallel} approximation for the photosphere and the equation of state of an ideal gas in {\it local thermodynamic equilibrium} (LTE). The structure of the atmosphere is described by a set of differential equations for the {\it hydrostatic} and {\it radiation} equilibrium.

\quad

London et al. \cite{LON86}  implemented the first detailed numerical calculation of atmospheric models  of a non-magnetic neutron star that included {\it Comptonization}, including {\it free-free} processes (due to ionized hydrogen and helium), and {\it bound-free absorption } (due to K shell transitions from Fe$^{+24}$ to Fe$^{+25}$), as sources to atmospheric {\it opacity}. In addition, they claimed that bound-free transitions were not important for bursts close to the Eddington limit.  They used {\it Kompaneets operator} to describe the Compton scattering in the non-relativistic and isotropic diffusion approximation, which was claimed to be adequate for hot neutron star model atmospheres with {\it effective temperatures below $\sim 2$ keV}.  Any other energy transport than radiation (\eg convection) was negligenced. They present an investigation of the radiative transfer of the neutron star photosphere for {\it 17  different models}, with {\it effective temperatures ranging from 0.25 to 3 keV}, {\it surface gravities $\log g$  = 14 and 15} (in cgs units),  and {\it helium and iron abundances (Fe/H) from 0 to 1 relative to solar abundance}. They assumed that the atmospheres were in a steady state and in radiative equilibrium, justified by the fact that the hydrodynamical and thermal time scales within the atmosphere are much shorter than the time scales over which the observed fluxes vary. They found that the spectrum significantly differed from a blackbody radiating at the effective temperature and the differences were associated to effects from {\it surface cooling}. With luminosity ratios (Eq. \ref{ll}) not exceeding 0.8, they obtained the color correction factors range {\it  $1.39 < f_{\rm c} < 1.76$}.


\quad 

In the following year, Ebisuzaki, \cite{EBI87} analytically described  the atmospheric structures of X-ray bursters, comparing his models to two observed sources, MXB 1636-536 and MXB 1608-522. He assumed that the spectrum deviates from a blackbody  due to Compton heating and cooling. He obtained relations for mass, radius and distance of the bursters, deriving two possible sets of these  parameters, taking into account the 4-1 keV absorption line, which would be due Cr {\rm \tiny XXIII} ($M=1.7-2.0 M_{\odot}, R = 11-12$ km and $D=6.3-6.7$ kpc) or due Fe {\rm \tiny XXV} ($M=1.8-2.1M_{\odot}, R=8-10$ km and $D=5.8-6.4$ kpc). In his calculations, he assumed $f_{\rm c} = 1.34$, $0.3 < l < 0.95$, and that the surface of the neutron star was covered with {\it pure helium matter} in the decay phases, without any heavy elements and neglecting any bound-free opacities.  His results revealed that at sufficient small $1-l$ an isothermal outer layer with $T>T_{\rm Eff}$ is formed in the photosphere due to Compton heating of electrons by hot photons coming from deeper layers.

\quad

Pavlov et al. \cite{PAV91} developed a numerical model of photosphere of X-ray bursts when the bursters are very close to the Eddington limit, with relative luminosities (Eq. \ref{ll}) ranging $l = 0.9-0.999$. They considered electron scattering and free-free transitions, assuming that the photosphere consists of hydrogen and helium. Their results showed a diluted spectrum for the burster atmosphere,  with spectral temperature coinciding to the electron temperature in an isothermal outer layer formed by Compton heating .  The  {\it dilution factor} was proportional to $f_{\rm c}^4$. 


\quad

In 2004, Madej et al. \cite{MAD04} presented 47 atmosphere models of  very hot neutron stars, where the  process of scattering of photons by free electrons was approximated by {\it coherent} and {\it isotropic} photoelectric collision. They used the integral description of Compton scattering employing an {\it angle-averaged redistribution function}.  The resulting flux spectrum of the outgoing radiation took into the account all the bound-free and  free-free monochromatic opacities relevant to the hydrogen-helium chemical compositions, \ie the effects of {\it Compton scattering} of radiation in the thermal plasma with fully relativistic thermal velocities. Their  hydrogen-helium model atmospheres and flux spectra were computed on a range of {\it effective temperature} from  $1 \times 10^7$ K to $3 \times 10^7$ K, and with only one value for surface gravity, $\log g = 15.0$. They found that {\it all color correction factors were consistently smaller than 1.9}. The group calculated values for the neutron star masses and radii from a direct fitting of the observed X-ray burst spectrum, proposing a best-fit for a fixed $\log g$ and  varying  $l$ (or $T_{eff}$) and the gravitational redshift $z$. The best-fit between all trial $\log g$ values would give the desired $g$ and $z$, and the mass and radii would be found by the equation \ref{m3}. However,  the curves on the $M-R$ planes to the fixed $\log g$ and $z$ reproduced very large uncertainties. In a following paper, \cite{MAJ05}, they adopted the same equations to calculate 106 models including  iron ions and adding  dozen bound-bound opacities for the highest ions of the iron. Their results differed significantly from pure hydrogen-helium spectra and from a the blackbody spectra, wit {\it color correction factors  $1.2<f_{\rm c} <1.85$}.


\quad

 Recently, Suleimanov et al. \cite{SPW10} adopted the {\it differential Kompaneets operators} to consider the Compton scattering in  360 atmospheric models (instead of the more general integral operator for the Compton scattering kernel).  They solved the radiation transfer equation and the hydrostatic equilibrium equation accounting for the radiation pressure by electron scattering. Their  models were computed for {\it six chemical compositions} (pure helium, pure hydrogen, solar hydrogen/helium mix with various heavy elements abundances, Z=1,0.3, 0.1 and 0.01 $Z_{\odot}$), and {\it three surface gravities} $\log g = 14.0, 14.3$, and $14.6$,   with  relative luminosities varying from $l=$0.001 to 0.98. They calculate the redshifts from $\log g$, by  adopting a neutron mass equal to $1.4 M_{\odot}$ for $\log g=14.0, 14.3, 14.6$ and getting $R=14.8, 10.88, 8.16$ and $z=0.18, 0.27,0.42$.  The emergent spectra of all the  models were redshifted and fitted by a diluted blackbody within 3-20 keV (RXTE/PCA range). With their large range of  color correction factors in terms of many luminosities, they suggest a new method for the determination of the neutron star radii and masses, at late phases of PRE X-ray bursts, based on the spectral (blackbody) normalization $K$, Eq. \ref{K} and  depending only on the color correction factor (Section \ref{sK}). Furthermore, in a following work \cite{SPW12}, the group extend their neutron star atmosphere models for 484 models (and $l$ up to $1.06-1.1$) using for the first time an {\it exact treatment of Compton scattering via the integral relativistic kinetic equation}.  In the new approach, the {\it Klein-Nishima reduction} in the electron scattering cross-section leaded to a decrease in the radiative acceleration relative to that for the Thomson scattering cross-section.  The values of the color correction factor for the new models with $l<0.8$ were almost identical to the old models based on Kompaneets operator. However, the comparison to  \cite{MAD04} showed dramatic differences in the spectral shape.

\quad


%%%%%%%%%%%%%%%%%%%%%%%%%%%%%%%%%%%%%%%%%%%%%%%%%%%%%%%%%%%%%%%%%%%%

\subsection{Fits for the Atmosphere Models}

Our study begun by analysing the new atmospheric models provided by the LANL group   and comparing them with  the same fits performed by  \cite{SPW10}\footnote{Suleimanov et al. have their atmosphere models available in the authors' website, \cite{SPWWEB}.}. These new sets of model atmosphere have their spectra based on {\it Thomson scattering} atmospheres ({\it gray and multigroup radiation diffusion} are included) with two modules: the radiation diffusion approximation, based on \cite{FLE71} and the full radiation transfer using Monte Carlo methods, based on \cite{WIN95}.

\quad  

The emergent spectra for X-ray burster's atmosphere models, for some set of input parameters (\eg  {\it relative luminosity}, {\it surface gravity}, and {\it chemical composition}), can  be fitted by a  diluted blackbody (Eq. \ref{dilbb}),
\begin{equation}
 F_{\rm Eff} \sim w B (f_{\rm c} T_{\rm Eff}),
 \label{fe}
\end{equation}
in some chosen energy band. For instance, \cite{SPW10}  limit their calculations to 3-20 keV, corresponding to the often used PCA detector of the RXTE observatory. The same choice was made in our  calculations.

\quad 


We developed {\it five different} ways to performance this  {\it non-linear least square fit},  with two free parameters, the  dilution term, $w$, and the color correction factor, $f_{\rm c}$:

\begin{enumerate}
 \item Minimizing the sum
 \begin{equation}
  \sum_{\rm n=1}^N(F_{\rm E_n} - w_1 B_{\rm E_n}(f_{\rm c,1}T_{\rm eff}))^2,
  \label{eqfit1}
 \end{equation}
   or
 \begin{equation*}
  \int_{\rm E_{\rm min}}^{E_{\rm max}}  (F_{\rm E_n} - w_1 B_{\rm E_n}(f_{\rm
c,1}T_{\rm eff}))^2
\frac{dE}{E},
 \end{equation*}
where $N$ is the number of photon energy points.

\quad


\item Minimizing
 \begin{equation*}
  \sum_{\rm n=1}^N  \frac{(F_{\rm E_n} - w_2 B_{\rm E_n}(f_{\rm c,2}T_{\rm
eff}))^2}{E_n^2},
 \end{equation*}
  or
 \begin{equation*}
  \int_{E_{\rm min}}^{E_{\rm max}}  (F_{\rm E_n} - w_2
B_{\rm E_n}(f_{\rm c,2}T_{\rm eff}))^2
\frac{dE}{E^3},
 \end{equation*}
 where we fit the photon count flux, not the energy flux.



\quad


 \item Minimizing
  \begin{equation*}
  \sum_{\rm n=1}^N  (F_{\rm E_n} - w_3 B_{\rm E_n}(f_{\rm c,3}T_{\rm
eff}))^2E_{\rm n},
 \end{equation*}
 or
 \begin{equation*}
  \int_{E_{\rm min}}^{E_{\rm max}}  (F_{\rm E_n} - w_3
B_{\rm E_n}(f_{\rm c,3}T_{\rm eff}))^2 dE.
 \end{equation*}
 
 
\quad


 \item Minimizing Eq. \ref{dilbb} as in the first method, but with only one free parameter, \ie $w_1 \longrightarrow
w_4=f^{-4}_{\rm c,4}$.
 
\quad

 \item Computing $f_{\rm c,5}$ by dividing the energy where the peak of the model flux $F_{\rm E}$ is reached by the peak  energy of the blackbody spectrum $B_{\rm E}(T_{\rm eff})$. Note that this color correction factor does not depend on the chosen energy band. This  procedure was proposed by \cite{MAD04} and \cite{MAJ05}.
\end{enumerate}


\quad

%%%%%%%%%%%%%%%%%%%%%%%%%%%%%%%%%%%%%%%%%%%%%%%%%%%%%%%%%%%%%%%%%%%%

\subsection{Preliminary Results for the  Fits}

The results of the five fits for the atmospheric models calculated by the LANL group and by Suleimanov et al. \cite{SPW10} can be seen in the Figs. \ref{plot1}, \ref{plot2}, and \ref{plot3}. They were written in \texttt{Python} \cite{py} with the \texttt{NumPy} libraries  \cite{numpy} and are available for reference in a git repository host by \cite{git}. 

\quad



 \begin{figure} [ht]
\includegraphics[scale=0.45]{figs/all-data.png}
\caption{Reproduction of the first fit (Eq. \ref{eqfit1}) for the 18 models (and 20 values of relative luminosities each)  from Suleimanov et al. \cite{SPW10} (Models 1-3: $X=1$, $Z=0$, and $\log g =14.0, 14.3, 14.6$; models 4-6: $X=0.74$, $Z=Z_{\odot}$, and $\log g =14.0, 14.3, 14.6$; models 7-9: $X=0.74$, $Z=0.30Z_{\odot}$, and $\log g =14.0, 14.3, 14.6$; models 10-12: $X=1$, $Z=0.1Z_{\odot}$ and $\log g =14.0, 14.3, 14.6$; models 13-15: $X=1$, $Z=0.01Z_{\odot}$ and $\log g =14.0, 14.3, 14.6$; models 16-18: $Y=1$, $Z=0$, and $\log g =14.0, 14.3, 14.6$).}
\label{plot1}
\end{figure} 



 \begin{figure} [ht]
\includegraphics[scale=0.45]{figs/models.png}
\caption{Reproduction of the first fit (Eq. \ref{eqfit1}) for  4 selected models (and 20 values of relative luminosities each) from Suleimanov et al \cite{SPW10} and LANL's data (model 4: $X=0.74$, $Z=Z_{\odot}$, $\log g =14.0$; model 7: $X=0.74$, $Z=0.30Z_{\odot}$,  $\log g =14.0$; model 10: $X=1$, $Z=0.1Z_{\odot}$, $\log g =14.0$; model 13: $X=1$, $Z=0.01Z_{\odot}$ and $\log g =14.0$).}
\label{plot2}
\end{figure} 




 \begin{figure} [ht]
\includegraphics[scale=0.45]{figs/all-fits2.png}
\includegraphics[scale=0.45]{figs/all-fits.png}
\caption{Comparison of all the five fit modules for two of the models from LANL's (in this case, model 4, with $X=0.74$, $Z=Z_{\odot}$ and $\log g =14.0$, and module 13, with $X=1$, $Z=0.01Z_{\odot}$ and $\log g =14.0$.}
\label{plot3}
\end{figure} 



\quad

The values for the color correction factors tend to be larger at luminosities close to the Eddington limit, and decrease when $l$ is small. This is due the {\it decreasing} role of Compton scattering, when comparing with the other sources of {\it opacity} \cite{SPW10}.


